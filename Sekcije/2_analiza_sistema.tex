\section{\bfseries Analiza sistema}

Osnovna svrha sistema je da omogući korisnicima aplikacije da lako i efikasno pronađu prevoz od jednog odredišta do drugog. Aplikacija unutar ovog projekta ograničiće se na gradski i međugradski prevoz unutar jedne države, tj. neće pružati usluge međunarodne vožnje. Aplikacija će, pored gorepomenute usluge, pružati mogućnost korisnicima da se lako informišu, prijave i obuče za vozače, koji će ubuduće drugim korisnicima pružati usluge i pritom biti plaćeni za svoj rad. Naša CarGo aplikacija kao primarni zadatak ima da pruži bezbednu vožnju i vozaču i putnicima. To omogućavaju razne mere predostrožnosti, poput redovnih provera vozača u vidu testova ličnosti, snalaženja u saobraćaju i njihovog poznavanja zakona, redovne mehaničke provere vozila, mogućnost ocenjivanja vozača, vozila, a i putnika, i blokada naloga u slučaju nezadovoljavajućih rezultata. Dodatnu zaštitu putnika i vozača pružaju sigurnosne kamere unutar i van vozila, koje se mogu iskoristiti za rešavanje mogućih sporova. Aplikacija takođe obezbeđuje visoku količinu transparentnosti u vidu unapred poznate cene prevoza. Na kraju, cilj aplikacije je da obezbedi da i vozač i putnici budu zadovoljni i da je koriste i ubuduće.

\subsection{\bfseries Dijagram konteksta}

\quad Na slici \ref{fig:CarGoContextDiagram} prikazani su dijagram konteksta i akteri, a na slici \ref{fig:dtp1} je dat dijagram toka podataka nivoa jedan.
\paragraph{Registrovanje korisnika:}
    Da bi korisnik mogao da se prijavi prvo mora da se registruje. Registraciju obavlja sam i dobija odgovor na kraju da li je uspešno registrovan ili ne.
\paragraph{Login korisnika:}
    Da bi korisnik mogao da zatraži vožnju mora biti prijavljen. On to radi samostalno i kao odgovor dobija da li se uspešno prijavio.
\paragraph{Rad sa vozačima:}
    Uslov da bi vozač mogao da vozi korisnike je da on bude registrovan. Proces registracije obavlja se uz pomoć službe za ljudske resurse. Takođe vozač, ukoliko nije pogodan, može biti obrisan iz sistema i samim tim će mu biti zabranjeno da dalje prevozi korisnike. Taj posao takođe obavlja služba za ljudske resurse. Pored toga, menadžer za ljudske resurse zadužen je i da napravi zahtev šefu voznog parka i da da vozilo na korišćenje vozaču, ukoliko novoregistrovani vozač nema svoje.
\paragraph{Login vozača:}
    Kako bi mogao da prevozi korisnike, vozač prvo mora da se prijavi. On to radi samostalno i kao odgovor dobija da li se uspešno prijavio.
\paragraph{Vožnja:}
    Kad korisnik napravi zahtev za vožnju, slobodan vozač može da ga prihvati. Vozač zatim dolazi na dogovorenu lokaciju i prevozi korisnika na željeno mesto. Kad stigne na željenu lokaciju, korisnik plaća vožnju i, ako želi, može da oceni vozača.
\paragraph{Nabavka vozila:}
    U sistemu postoje dve vrste vozača. Vozači koji imaju svoj automobil i oni koji dobijaju automobil na korišćenje prilikom registracije. Da se ne bi desilo da nema slobodnih automobila za novoregistrovanog vozača, šef voznog parka obavlja nabavku novih automobila.

\begin{figure}[H]
\begin{center}
\includegraphics[width=\textwidth]{CarGoContextDiagram.png}
\end{center}
    \caption{Dijagram konteksta infomacionog sistema}
\label{fig:CarGoContextDiagram}
\end{figure}

\begin{figure}[H]
\begin{center}
\includegraphics[width=\textwidth]{ISzaCarGo.png}
\end{center}
    \caption{Dijagram toka podataka nivoa 1}
\label{fig:dtp1}
\end{figure}

\subsection{\bfseries Akteri}
\begin{itemize}
    \item Korisnici ovog sistema su svi oni kojima je potreban transport od jednog odredišta do drugog. Možemo ih podeliti na fizička i pravna lica. 
    \begin{itemize}
        \item Fizička lica - osobe koje usluge ovog sistema koriste preko svojih računa
        \item Pravna lica - osobe koje usluge ovog sistema koriste preko računa firme u kojoj rade
    \end{itemize}
    \item Vozači su ljudi kojima ovaj informacioni sistem posreduje kako bi pružali usluge korisnicima kojima je potreban prevoz. Vozači većinski koriste svoje automobile za prevoz.
    \item Banka je posrednik u transakciji između korisnika i vozača nakon izvršene usluge.
    \item Dobavljači vozila su grupa ljudi koja se bavi nabavkom vozila koja se koriste za prevoz korisnika.
    \item Služba za ljudske resurse je grupa ljudi koja odlučuje ko je podoban da bude vozač, odnosno ko je bezbedan po korisnike sistema, i ima dozvolu za vozilo kojim upravlja. Ukoliko vozač nema svoje vozilo, služba je dužna da o tome obavesti šefa voznog parka.
     \item Šef voznog parka je osoba koja sastavlja porudžbine o broju potrebnih vozila, i šalje te porudžbine dobavljačima.
\end{itemize}