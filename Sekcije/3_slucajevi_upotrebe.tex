\section{\bfseries Slu\v cajevi upotrebe}

\subsection{\bfseries Registrovanje korisnika}
\subsection{\bfseries Login korisnika}


\subsection{\bfseries Rad sa vozačima}
\subsubsection{\bfseries Registrovanje vozača}

\begin{itemize}
	\item Kratak opis:
		\begin{itemize}
			\item Korisnik pristupa veb stranici i prijavljuje se za pružanje usluga na putu
		\end{itemize}
 
	\item Ucesnici:
		\begin{itemize}
			\item Zainteresovana osoba koja želi da postane deo CarGo zajendice		
		\end{itemize}				

	\item Preduslovi:
		\begin{itemize}
		    \item Prijavljeni mora da ima vozačku dozvolu
		    \item Bar 5 godina iskustva u vožnji
		    \item Napredno poznavanje grada
		    \item Auto koji nije stariji od 10 godina
		    \item Pametan telefon
		    \item Da nije osuđivan
		    \item Uspešno položen test ličnosti
		\end{itemize}

	\item Postuslovi:
		\begin{itemize}
			\item Korisnik je registrovan kao vozač
		\end{itemize}		


	\item Glavni tok:
		\begin{itemize}
		    \item Korisnik odlazi na veb stranicu i nalazi formu za prijavu
		    \item Korisnik popunjava prijavu
		    \item Na mejl stize potvrda o uspešnosti prijavljivanja i termin dolaska na razgovor sa HR
		    \item Potencijalni vozač donosi na razgovor potrebnu dokumentaciju i radi test
		    \item HR odlučuje da li je vozač kompetentan za tu poziciju i ukoliko jeste sistem beleži novog vozača u bazu
		\end{itemize}

	\item Alternativni tok:
		\begin{itemize}
    			\item Korisnik nije uneo ispravne podatke za prijavu. Slučaj upotrebe se nastavlja na drugom koraku glavnog toka
		    \item Korisnik nije doneo potrebnu dokumentaciju na razgovor. U tom slučaju korisnik dobija novi termin za razgovor
		\end{itemize}


	\item Dodatne informacije:
		\begin{itemize}
			\item Neophodni podaci za prijavu vozača su ime i prezime, validna e-mail adresa, broj telefona, godina registracije, marka i tip vozila pod uslovom da ima vozilo i za koji grad se prijavljuje
		\end{itemize}						


\end{itemize}


\subsubsection{\bfseries Zahtev za automobil}
\begin{itemize}
	\item Kratak opis:
		\begin{itemize}
			\item Korisnicima koji nemaju sopstveni automobil potrebno je obezbediti prevozno sredstvo		
		\end{itemize}

	\item Učesnici:
		\begin{itemize}
		    \item HR
		    \item Šef voznog parka
		    \item Vozač
		\end{itemize}


	\item Preduslovi:
		\begin{itemize}
		    \item Vozač je registrovan
		    \item U prijavi je označeno da je potrebno vozilo
		\end{itemize}


	\item Postuslovi:
		\begin{itemize}
			\item Vozač dobija vozilo
	\end{itemize}

	\item Glavni tok:
		\begin{itemize}
		    \item Ukoliko vozač nema vozilo HR salje zahtev šefu voznog parka
		    \item Šef voznog parka proverava da li ima slobodnih vozila
		    \item Ukoliko ima dodeljuje vozilo vozaču
		\end{itemize}

	\item Alternativni tok:
		\begin{itemize}
		    \item Ukoliko nema, vozač se stavlja na čekanje i prelazi se na slucaj upotrebe nabavke vozila
		\end{itemize}

	\item Dodatne informacije:
		\begin{itemize}
			\item Na osnovu prijave koju je dostavio vozač HR zna da li on ima svoje vozilo ili je zatražio njihovo
		\end{itemize}

\end{itemize}


\subsubsection{\bfseries Otpuštanje vozača}
%TODO


\subsection{\bfseries Login vozača}

\subsection{\bfseries Vo\v znja}

\subsubsection{\bfseries Naru\v civanje vo\v znje}
\noindent Slučaj upotrebe: Naručivanje vožnje\\
1. Kratak opis:
\par -Korisnik naručuje vožnju radi transporta sa jednog odredišta na drugo. \\
2.  Učesnici:  
\par - Korisnik
\par - Vozač \\
3. Preduslovi: 
\par - Korisnik mora na svom telefonu posedovati aplikaciju.
\par - Korisnik mora imati dovoljno novca na računu da bi mogao da plati vožnju. \\
4. Postuslovi:
\par - Mora postojati slobodan vozač koji će moći da izvrši transport od jedne lokacije do druge.\\
5. Osnovni tok:
\par 1. Korisnik se registruje
\par 2.  Definise putem svoje aplikacije početnu lokaciju
\par 3. Definiše lokaciju na koju želi da bude transportovan.
\par 4. Zahtev se šalje serveru
\par 5. Server šalje zahtev vozačima
\par 6. Vozači koji žele i u mogućnosti su da prime vožnju to i čine
\par 7. Korisnik bira  nekog od ponuđenih vozača u zavisnosti od nekoliko parametara(ocena vozača, udaljenost od trenutne lokacije) ukoliko mu cena odgovara.
\par 8. Korisnik čeka odabranog vozača na definisanoj lokaciji u cilju transporta. \\

\noindent 6. Alternativni tok:
\par 7: Ne postoji slobodno vozilo koje može izvršiti transport korisnika servisa ili koje zadovoljava želje korisnika:
\par - Korisnik se obaveštava da trenutno ne postoji slobodno vozilo i ukoliko to korisnik želi, stavlja se na listu čekanja dok se ne oslobodi neko vozilo.

\subsubsection{\bfseries Prihvatanje vo\v znje}
\noindent Slučaj upotrebe: Prihvatanje vožnje \\
1. Kratak opis:
\par Vozač prihvata ili odbija zahtev za prevoz korisnika. \\
2.  Učesnici:  
\par - Vozač \\
3. Preduslovi: 
\par - Zahtev za prevoz poslat od strane korisnika \\
4. Postuslovi:
\par - Vozač je prihvatio vožnju ukoliko je to želeo \\
5. Osnovni tok:
\par 1. Korisnik šalje zahtev za prevoz
\par 2.  Zahtev preko servera stiže do vozača
\par 3.  Vozač prihvata ili odbija korisnički zahtev \\

\subsubsection{\bfseries Prevoz putnika}
%in progress

\subsubsection{\bfseries Ocenjivanje voza\v ca}
\noindent Slučaj upotrebe: Ocenjivanje vozača \\
1. Kratak opis: 
\par Korisnik ocenjuje vozača na osnovu utisaka koji je isti na njega ostavio tokom vožnje. \\
2.  Učesnici:  
\par - Korisnik \\
3. Preduslovi: 
\par - Korisnik prevežen od jedne lokacije do druge. \\
4. Postuslovi:
\par - Vozač je ocenjen ukoliko je korisnik izabrao tu opciju \\
5. Osnovni tok:
\par 1. Korisnik je prevežen sa jedne lokacije na drugu
\par 2.  Korisnik bira da li hoće da oceni vozača
\par 3.  Korisnik vrši ocenjivanje vozača ocenom od 1 do 5 \\

\noindent 6. Alternativni tok:
\par 2: Ukoliko je korisnik odabrao opciju da ne želi da oceni vozača korak 3 se preskače.

\subsubsection{\bfseries Naplata vo\v znje}
\noindent Slučaj upotrebe: Plaćanje vožnje \\
1. Kratak opis:
\par Korisnik isplaćuje uslugu prevoza. \\
2.  Učesnici:  
\par - Korisnik
\par - Banka
\par - Vozač \\
3. Preduslovi: 
\par - Neophodno je da korisnik naruči vožnju. \\
4. Postuslovi:
\par - Uplata je uspešno legla vozaču. \\
5. Osnovni tok:
\par 1. Korisnik naručuje vožnju
\par 2.  U toku naručivanja vožnje, korisnik vrši uplatu novca
\par 3.  Novac se uplaćuje preko aplikacije
\par 4. Banka vrši prosleđivanje uplaćenog novca  vozaču
\par 5. Vozaču leže uplata \\

\noindent 6. Alternativni tok:
\par 3: Novac se uplaćuje direktno vozaču
\par - Tok se nastavlja normalno preskakanjem 4. koraka

\newpage

\subsection{\bfseries Nabavka vozila}

\subsubsection{\bfseries Kreiranje narudžbine vozila}
\noindent Slučaj upotrebe: Kreiranje narudžbine vozila:\\
1. Kratak opis:
\par -Šef voznog parka sastavlja porudžbinu o broju vozila pomoću informacije o vozačima kojima su potrebna vozila, a koju će kasnije proslediti dobavljaču vozila.\\
2. Učesnici:
\par -Šef voznog parka
\par -HR tim\\
3. Preduslovi:
\par -Postoje vozači kojima su potrebna vozila.\\
4. Postuslovi:
\par -Porudžbina je sastavljena i spremna da bude poslata.\\
5. Osnovni tok:
\par 1. HR tim šalje dopis šefu voznog parka na nedeljnom nivou koliko novih vozača nema svoje vozilo.
\par 2. Šef voznog parka proverava da li ima 10 ili više vozača kojima je potrebno vozilo ili postoji barem jedan vozač koji čeka na vozilo više od dve nedelje.
\par 3. Šef voznog parka sastavlja porudžbinu.\\

\noindent 6. Alternativni tok:
\par 2: Ima manje od 10 vozača kojima je potrebno vozilo ili nijedan vozač ne čeka 2 nedelje.
\par -Šef voznog parka čeka sledeći dopis od HR tima i kreće ponovo od koraka 2.

\subsubsection{\bfseries Nabavka vozila}
\noindent Slučaj upotrebe: Nabavka vozila\\
1. Kratak opis:
\par -Šef voznog parka ima zadatak da naruči potrebnu količinu vozila kako bi ih prosledio novozaposlenim vozačima koji nemaju svoja vozila.\\
2. Učesnici:
\par -Dobavljač
\par -Šef voznog parka\\
3. Preduslovi:
\par -Postoji barem jedan vozač koji nema svoje vozilo.\\
4. Postuslovi:
\par -Pribavljeno je onoliko vozila koliko ima vozača koji nemaju svoja.\\
5. Osnovni tok:
\par 1. Šef voznog parka proverava da li postoji neki vozač koji je zaposlen i čeka na firmu da mu pribavi vozilo.
\par 2. Šef voznog parka nakon provere sastavlja porudžbinu.
\par 3. Šef voznog parka stupa u kontakt sa dobavljačem.
\par 4. Šef voznog parka isporučuje dobavljaču zahtevan broj vozila.
\par 5. Dobavljač prihvata porudžbinu.
\par 6. Dobavljač isporučuje Šefu voznog parka zahtevan broj vozila.
\par 7. Šef voznog pa:rka ih smešta u vozni park do raspodele vozila vozačima.\\

\noindent 6. Alternativni tok:
	\par 5: U slučaju da dobavljač nije u stanu da ostvari porudžbinu
	\par	-Šef voznog parka odlaže nabavku u slučaju da su vozila potrebna za manje od 10 vozača.
	\par	-Šef voznog parka pronalazi drugog dobavljača u slučaju da postoji 10 ili više vozača koji čekaju na vozila i nastavlja do koraka 5.

\subsubsection{\bfseries Predaja vozila vozaču}
\noindent Slučaj upotrebe: Predaja vozila vozaču\\
1. Kratak opis: 
\par -Šef voznog parka prosleđuje vozilo iz voznog parka onom vozaču koji se zaposlio a nema svoje vozilo.\\
2. Učesnici:
\par -Šef voznog parka
\par -Vozač\\
3. Preduslovi:
\par -Vozač koji nema svoje vozilo pa čeka na firmino vozilo\\
4. Postuslovi:
\par -Vozaču je predato vozilo na korišćenje.\\
5. Osnovni tok:
\par 1. Šef voznog parka obaveštava vozača da li ima vozilo.
\par 2. Vozač i šef voznog parka se dogovaraju kada će se sastati.
\par 3. Vozač i šef voznog parka se nalaze.
\par 4. Vozač napismeno prihvata odgovornost za to vozilo.
\par 5. Vozač preuzima vozilo.\\

\noindent 6. Alternativni tok:
	\par 1: Šef voznog parka nema vozilo za vozača
	\par -Šef voznog parka dodaje vozača na spisak vozača koji čekaju na nabavku vozila nakon koje se kreće od koraka 1.
	
\subsubsection{\bfseries Upravljanje podacima o vozačima}

\subsubsection{\bfseries Upravljanje podacima o dobavljačima}